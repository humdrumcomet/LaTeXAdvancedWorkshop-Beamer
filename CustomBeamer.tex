% arara: pdflatex: { shell: yes }
% arara: makeglossaries
% arara: biber
% arara: pdflatex: { shell: yes }
% arara: pdflatex: { shell: yes }


% Demo report and presentation prepared by Aaron English (humdrumcomet on github)
\documentclass{beamer}

\usetheme{Marburg}
% \usecolortheme{owl}

\usepackage{geometry} % useful for defining page geometries
\usepackage{hyperref} % used for creating hyperlinks in documents. Both to the web and within the document itself
\usepackage[utf8]{inputenc} % how to treat the written file (as utf8)
\usepackage[T1]{fontenc} % the encoding for the output file T1 is the most common, includes accents and many other commonly needed/used characters
\usepackage[style=ieee,backend=biber]{biblatex} % for handling bibliographies
\usepackage{float} % added control over
\usepackage{graphicx} % tools for inclusion of graphics
\usepackage{booktabs} % adding commands to improve the look of tables
\usepackage{csvsimple} % simplify table creation by importing .csv files directly
\usepackage{siunitx} % consistent notation and correct formatting of units
\DeclareSIUnit{\torr}{Torr} % Custom unit definition
\usepackage{minted} % inclusion of code blocks with syntax highlighting and
\setminted{linenos, autogobble, fontsize=\footnotesize, breaklines=true} % set global options for minted environments
\usepackage{chemformula} % for writing chemical formulae
\usepackage[useregional]{datetime2}

%%%%%%%%%%%%%%%%%%%%%%%%%%%%%%%%%%%%%%%%%%%%%%%%%%%%%%%%%%%%%%%%%%%%%%%%%%%%%%%%%%%%%%%%%%%%%%%%%%%%%%%%%%%%%%%%%%
% New packages introduced in this talk
\usepackage{standalone}
\usepackage{import}
\usepackage{catchfilebetweentags}
\usepackage[debug, toc, section=section, acronym, symbols]{glossaries} % Glossaries package
\usepackage{tikz}
    \usetikzlibrary{math, arrows, circuits.ee.IEC, positioning, shapes.arrows, shapes.geometric, external}
\usepackage{pgfplots}
    \pgfplotsset{compat=newest, compat/show suggested version=false}
    \usepgfplotslibrary{groupplots}
\usepackage[siunitx,american voltages, american currents, RPvoltages]{circuitikz}

% A command for less cramped nested fractions
\newcommand\ddfrac[2]{\frac{\displaystyle #1}{\displaystyle #2}}

%Get rounded wire jumps
\tikzset{
    declare function={% in case of CVS which switches the arguments of atan2
        atan3(\a,\b)=ifthenelse(atan2(0,1)==90, atan2(\a,\b), atan2(\b,\a));},
        kinky cross radius/.initial=+.125cm,
        @kinky cross/.initial=+, kinky crosses/.is choice,
        kinky crosses/left/.style={@kinky cross=-},kinky crosses/right/.style={@kinky cross=+},
        kinky cross/.style args={(#1)--(#2)}{
        to path={
          let \p{@kc@}=($(\tikztotarget)-(\tikztostart)$),
              \n{@kc@}={atan3(\p{@kc@})+180} in
          -- ($(intersection of \tikztostart--{\tikztotarget} and #1--#2)!%
                 \pgfkeysvalueof{/tikz/kinky cross radius}!(\tikztostart)$)
          arc [ radius     =\pgfkeysvalueof{/tikz/kinky cross radius},
                start angle=\n{@kc@},
                delta angle=\pgfkeysvalueof{/tikz/@kinky cross}180 ]
          -- (\tikztotarget)}}
      }

% SI unit pkg conf
\sisetup{
    load-configurations = abbreviations,
    detect-family = true,
    per-mode = reciprocal
}%
\DeclareSIUnit{\torr}{Torr}

% Bibliography .bib file locations should maybe be local?
\addbibresource[location=local]{./accelerators.bib}

% Add Constants list using glossary
\newglossary[cgls]{constants}{cstog}{cstig}{Constants}
% Alphabetize glossary and acronyms list
\makeglossaries

%%% Constants
%%%%%%%%%%%%%%%%%%%%%%%%%%%%%%%%%%%%%%%%%%%%%%%%%%%%%%%%%%%%%%%%%%%%%%%%%%%%%%%%%%%%%%%%
\newglossaryentry{c}
{
    type=constants,
    name={\ensuremath{c}},
    description={\mbox{} speed of light (\SI{2.997924e8}{\meter\per\second})}
}
\newglossaryentry{boltz}
{
    type=constants,
    name={\ensuremath{k_{B}}},
    description={\mbox{} Boltzmann constant (\SI{8.617330e-5}{\electronvolt\per\kelvin})}
}
\newglossaryentry{eps0}
{
    type=constants,
    name={\ensuremath{\varepsilon_{0}}},
    description={\mbox{} permittivity of free space (\SI{8.854187e-12}{\farad\per\meter})}
}
\newglossaryentry{eq}
{
    type=constants,
    name={\ensuremath{e}},
    description={\mbox{} elementary charge (\SI{1.602176e-19}{\coulomb})}
}

%%% Symbols
%%%%%%%%%%%%%%%%%%%%%%%%%%%%%%%%%%%%%%%%%%%%%%%%%%%%%%%%%%%%%%%%%%%%%%%%%%%%%%%%%%%%%%%%
\newglossaryentry{mg}
{
    type=symbols,
    name={\ensuremath{G}},
    description={Mosfet Gate}
}
\newglossaryentry{md}
{
    type=symbols,
    name={\ensuremath{D}},
    description={Mosfet Drain}
}
\newglossaryentry{ms}
{
    type=symbols,
    name={\ensuremath{S}},
    description={Mosfet Source}
}
\newglossaryentry{press}
{
    type=symbols,
    name={\ensuremath{P}},
    description={Pressure (units of \si{\torr})}
}
\newglossaryentry{time}
{
    type=symbols,
    name={\ensuremath{t}},
    description={time (units of minutes)}
}
\newglossaryentry{ener}
{
    type=symbols,
    name={\ensuremath{E}},
    description={Energy}
}
\newglossaryentry{efermi}
{
    type=symbols,
    name={\ensuremath{E_{f}}},
    description={Fermi Energy of a Material}
}
\newglossaryentry{fermiDist}
{
    type=symbols,
    name={\ensuremath{f(E)}},
    description={The Fermi-Dirac distribution}
}
\newglossaryentry{econd}
{
    type=symbols,
    name={\ensuremath{E_{C}}},
    description={Conduction band energy level}
}
\newglossaryentry{evalence}
{
    type=symbols,
    name={\ensuremath{E_{V}}},
    description={Valence band energy level}
}
\newglossaryentry{egap}
{
    type=symbols,
    name={\ensuremath{E_{G}}},
    description={Bandgap}
}
\newglossaryentry{temp}
{
    type=symbols,
    name={\ensuremath{T}},
    description={Temperature (\si{\kelvin})}
}
\newglossaryentry{rd}
{
    type=symbols,
    name={\ensuremath{R_{D}}},
    description={Resistance}
}
\newglossaryentry{vout}
{
    type=symbols,
    name={\ensuremath{V_{out}}},
    description={Output voltage}
}
\newglossaryentry{vin}
{
    type=symbols,
    name={\ensuremath{V_{in}}},
    description={Input voltage}
}

%%%%%%%%%%%%%%%%%%%
%%%%%%%%%%%%%%%% Acronym Only
\newglossaryentry{ndfeb}
{
    type=\acronymtype,
    name={NdFeB},
    description={Neodymium Iron Boron Rare Earth Permanent Magnet},
    first={Neodymium iron boron (NdFeB)}
}
\newglossaryentry{smco}
{
    type=\acronymtype,
    name={SmCo},
    description={Samarium Cobalt Rare Earth Permanent Magnet},
    first={Samarium cobalt (SmCo)}
}
\newglossaryentry{cnc}
{
    type=\acronymtype,
    name={CNC},
    description={Computer Numerically Controlled},
    first={Computer Numerically Controlled (CNC)}
}
\newglossaryentry{ptfe}
{
    type=\acronymtype,
    name={PTFE},
    description={Polytetraflouroethylene (PTFE), commonly referred to by the brand name Teflon, is a synthetic polymer with highly desirable electrical breakdown characteristics},
    first={Polytetraflouroethylene (PTFE)}
}

%%%%%%%%%%%%%%%%%%%
%%%%%%%%%%%%%%%% Glossary Only
\newglossaryentry{cpp}
{%
    name={C++},
    description={C++ is a programming language that can be used as an object oriented programming language, an imperative programming language, and still provide low-level memory control. Note: All C++ code used in this work is compiled under the C++11 standard}
}
\newglossaryentry{python}
{%
    name={Python},
    description={Python is a flexible scripting language that can be used as an imperative language, an object oriented language. Note: All python code used in this work is written and executed using Python3.6}
}
\newglossaryentry{ibsimu}
{%
    name={IBSIMU},
    description={IBSIMU is a \gls{cpp} library for ion optics, space-charge calculation, and plasma extraction. It is able to import 2-D and 3-D geometries as exported in .STL or .DXF file formats. It also allows for mathematical definition of geometries. Note: Version 1.0.6 of the IBSIMU library was used for this work \cite{acc:14}}
}
\newglossaryentry{ansys}
{
    name={ANSYS},
    description={ANSYS Multiphysics is a software suite for multiphysics modelling and simulation}
}

%%%%%%%%%%%%%%%%%%%
%%%%%%%%%%%%%%%% Glossary and Acronym
\newglossaryentry{hvag}
{%
 name={HV},
 description={High Vacuum (HV) is a term which refers to vacuum pressures in the range of \SI{1e-3}{\torr} to \SI{1e-8}{\torr}\cite{acc:13}}
}
\newglossaryentry{hva}
{%
 type=\acronymtype,
 name={HV},
 description={High Vacuum},
 first={High Vacuum (HV)\glsadd{hvag}},
 see=[Glossary:]{hvag}
}

\title{Hello Beamer!}
\subtitle{Introducing Presentations in \LaTeX}
% \date{\DTMdisplaydate{2022}{10}{11}}
\author{Ghassan Arnouk \and
        Alec Bales D'Cruze \and
        Aaron English
}

\begin{document}
    \maketitle
    \begin{frame}[noframenumbering, plain]
        \frametitle{Contents}
        \tableofcontents
    \end{frame}
    \section{Intro}\label{sct:intro}
        \begin{frame}
            \frametitle{All of Your Favourites Work Like You Expect}
            \begin{itemize}
                \item Equations
                \item Tables
                \item Figures
                \item Glossaries
                \item Citations
                \item Automation
            \end{itemize}
        \end{frame}
    \section{Builtin Beamer Features}
        \begin{frame}
            \frametitle{Beamer Has Many Possible Themes}
            Many Beamer Themes Have:
            \begin{itemize}
                \item<1-> Progress bars
                \item<2-> Navigation links
                \item<3-> Other neat features
            \end{itemize}
        \end{frame}
        \begin{frame}
            \frametitle{Even Revealing Lists and Images}
            \begin{columns}[T]
                \begin{column}{0.5\textwidth}
                    \begin{enumerate}
                        \item<1-> Designed vacuum chamber
                        \item<2-> Designed modular support system
                        \item<3-> Designed plasma source and ion lens
                        \item<4-> Manufactured internal components
                        \item<5-> Assembled system
                        \item<6-> Tested system
                    \end{enumerate}
                \end{column}
                \begin{column}{0.5\textwidth}
                    \begin{figure}[htbp]
                        \begin{centering}
                            \begin{overlayarea}{\textwidth}{\textheight}
                                \only<1>{\includegraphics[width=\textwidth]{./Images/NewChamberAssembly/beamLineNoCartView2Cropped.jpg}}%
                                \only<2>{\includegraphics[width=\textwidth]{./Images/NewChamberAssembly/FullChambAssCrop.jpg}}%
                                \only<3>{\includegraphics[width=\textwidth]{./Images/NewChamberAssembly/PIGCropped.jpg}}%
                                \only<4>{\includegraphics[width=\textwidth]{./Images/IRLimages/manedPigPartsSmall.jpg}}%
                                \only<5>{\includegraphics[height=0.65\textheight]{./Images/IRLimages/pigWithEinz.jpg}}%
                                \only<6>{\includegraphics[width=\textwidth]{./Images/IRLimages/nitrogenPlasmaBeam.jpg}}%
                            \end{overlayarea}
                        \end{centering}
                    \end{figure}
                \end{column}
            \end{columns}
        \end{frame}
    \section{Demos}
        \begin{frame}
            \frametitle{A Couple of Equations}
            \begin{equation}
                \gls{fermiDist} = \left(e^{\left(\ddfrac{\gls{ener}-\gls{efermi}}{\gls{boltz}\gls{temp}}\right)}+1\right)^{-1}
                \label{eqt:fermiDirac}
            \end{equation}
            \begin{align}
                \frac{\text{d}x}{\text{d}t} &= \sigma (y-x)\\
                \frac{\text{d}y}{\text{d}t} &= x(\rho -z)-y\\
                \frac{\text{d}z}{\text{d}t} &= xy-\beta z
            \end{align}
        \end{frame}
        \begin{frame}
            \frametitle{A Table}
            \begin{table}[H]
                \centering
                \caption{Table of specified parameters and achieved values}
                \vspace{0.1cm}
                    \csvreader[
                    head to column names,
                    tabular = cccc,
                    table head = \toprule \bfseries Parameter & \bfseries Target & \bfseries Calculated & \bfseries Simulated \\\midrule,
                    table foot = \bottomrule,
                    ]{./Data/TargetsAndVals.csv}{}{\csvcoli & \csvcolii & \csvcoliii & \csvcoliv}
                \label{Table:Parameters}
            \end{table}
        \end{frame}
        \begin{frame}
            \frametitle{Some Citations, Glossary Entries, and Acronyms}
            \begin{itemize}
                \item This statement should be references \cite{acc:1}
                \item Talking about \gls{hva}
                \item Once more, \gls{hva}
            \end{itemize}
        \end{frame}
        \begin{frame}
            \frametitle{Bibliography}
            \printbibliography
        \end{frame}
        \begin{frame}
            \frametitle{Glossary}
            \printglossaries
        \end{frame}
    \section{TikZ}
        \begin{frame}
            \frametitle{TikZ Graphics}
            \begin{figure}[htbp]
                \begin{centering}
                    \begin{tikzpicture}[scale=0.8]
                            \draw[->] (0,0)--(0,6.5)node[left]{\gls{ener}};
                            \draw[->] (0,0)--(5.5,0)node[right]{\gls{fermiDist}};
                            \draw[dashed] (2.5,0)node[below]{$\sfrac{1}{2}$}--(2.5,3);
                            \draw[dashed] (0,3)node[left]{\gls{efermi}}--(2.5,3);
                            \draw (0.05,6)--(0.05,4.5);
                            \draw (0.05,4.5) to [out=270, in=90] (5,1.5);
                            \draw (5,1.5)--(5,0)node[below]{$1$};
                            \draw (0.05,6)--(0.05,5.5);
                            \draw (0.05,5.5) to [out=270, in=90] (5, 0.5);
                            \draw (5,0.5)--(5,0);
                            \draw[->] (0.3,3.3)--(2,4.5)node[right]{$\gls{temp}_{increasing}$};
                        \end{tikzpicture}
                    \caption{A TikZ Diagram showing a sketch of the Fermi-Dirac distribution}
                    \label{fig:tikzFDDist}
                \end{centering}
            \end{figure}
        \end{frame}
        \begin{frame}
            \frametitle{TikZ Plots}
            \begin{figure}[htbp]
                \begin{centering}
                    \begin{tikzpicture}[scale=0.8]
                        \begin{axis}[ymode=log,
                            grid=both,
                            ymin=1e-6,
                            ymax=1e3,
                            max space between ticks=30,
                            xmin=0,
                            xmax=350,
                            xlabel=\gls{time},
                            ylabel=\gls{press},
                            title=Chamber Pressure over Time]
                            \addplot[color=red, mark size=0.4, style=thick]table[x=time, y=p1, col sep=comma]{./Data/vacuumTech.csv};
                            \addplot[color=black, mark=*, mark size=0.4, style=thick]table[x=time, y=p2, col sep=comma]{./Data/vacuumTech.csv};
                        \end{axis}
                    \end{tikzpicture}
                    \caption{A TikZ plot from some data in a CSV file}
                    \label{fig:tikzPlot}
                \end{centering}
            \end{figure}
        \end{frame}
    
\end{document}
